\chapter{The Writings of Nephi}

\section{Question}
\Paragraph{How did Nephi gain a great knowledge of the goodness and mysteries of God?}

\begin{description}
  \item [Learned from others] Nephi learned from others. For example, he learned from his parents examples, such as in 1 Nephi 1:18-19 where Lehi taught repentence to the wicked people of Jerusalem and 1 Nephi 2:2-3 where Lehi was obedient to the Lord's command to leave Jerusalem. Note, Laman and Lemuel experienced the same things but did not learn what Nephi learned from them.
  \item [Had great desires to know] Nephi sought sincerely to know the truth. He prayed to God and God visited him and softened his heart so that he could believe.\footnote{Not all who desire to see and hear great things are given that privilage, as Jesus explained to his disciples "For verily I say unto you, that many prophets and righteous men have desired to see those things which ye see, and have not seen them; and to hear those things which ye hear, and have not heard them. (Matthew 13:17)}
\end{itemize}

\section{1 Nephi 1}

\paragraph{1 Nephi 1:18} Nephi recounts that the Lord showed Lehi `many great and marvelous things' but that they concerned something that doesn't seem very marvelous: the destruction of Jerusalem. So, there has to be more to what Lehi saw than just the destruction.

The things that Lehi saw and heard caused him to go and preach and prophesy to the people. Maybe the Lord asked him to go and teach about the things which he saw. (yes, see 1 Nephi 2:1)

\paragraph{1 Nephi 1:19} I think verse 19 contains the answer to my question. Nephi says that the things Lehi read in the book `manifested plainly of the coming of a Messiah, and also the redemption of the world.'

I think it is interesting that Nephi describes the people who mocked Lehi as `the Jews.' It is as if he is using that phrase to describe a sub-set of the people in and around Jerusalem at the time because he and his family were also `Jews.'
