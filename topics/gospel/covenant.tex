\chapter{Covenant}


\section{Covenants Made}

Covenants were made in these chapters:
\begin{itemize}
  \item Moses 5:4-9 - God's covenant with Adam and Eve.
  \item Genesis 6:18, 8:20-22, 9:8-17 - God covenants with Noah (The JST has some important additions).
  \item Genesis 22:15-18 - God covenants with Abraham (after Abraham was willing to sacrifice Isaac).
  \item Genesis 26:1-5 - God covenants with Isaac.
  \item Genesis 28:10-15 - God covenants with Jacob.
  \item Exodus 19-20 - God gives the covenant to Israel at Mt. Sinai (Mt. Horeb).
  \subitem Deuteronomy 5 - Moses recalls and explains the covenant administered at Mt. Sinai (Mt. Horeb).
  \item Deuteronomy 29 - Moses administers a covenant to Israel in Moab.
  \item 2 Nephi 2:5-7 - God's gives Lehi and his descendants this land by covenant.
  \subitem God also gave it to whoever He leads to this land.
  \subitem No one will come to this land except they are led by God.
  \item 2 Nephi 3:5-21 - covenant with Joseph (Jacob's son)
\end{itemize}

Other significant chapters related to God's covenant with His children:
\begin{itemize}
  \item 1 Nephi 22 - Nephi explains God's covenant promises from Isaiah 48 and 49.
  \item 3 Nephi 20 - the Savior explains how He will fulfill the covenant before His second coming.
\end{itemize}

\section{All are equal before God}

\section{God's Promises}

\begin{quotation}
I will be his father, and he shall be my son. If he commit iniquity, I will chasten him with the rod of men, and with the stripes of the children of men: But my mercy\footnote{also: love (NIV), favor (NLT), loving devotion (BSB))} shall not depart away from him, as I took it from Saul, whom I put away before thee" (2 Samuel 7:14-15)
\end{quotation}

God's promise to save His children centers on a Redeemer and Savior. In Isaiah 49 (and 1 Nephi 21) God reassures his scattered and captive covenant people that He will provide a Savior (the servant in verses 1-4) who will be commissioned to not only "bring Jacob again to him"\footnote{Isaiah 49:5} but to also be "a light to the Gentiles" and a means to bring "salvation unto the ends of the earth."\footnote{Isaiah 49:6}

\begin{quotation}
Thou art my servant, O Israel, \textit{in whom I will be glorified} (Isaiah 49:3).\index{covenant, Savior}
\end{quotation}

\begin{quotation}
Thus saith the Lord: In an acceptable time have I heard thee, O isles of the sea, and in a day of salvation have I helped thee; and I will preserve thee, and give thee my servant for a covenant of the people, to establish the earth, to cause to inherit the desolate heritages;

That thou mayest say to the prisoners: Go forth; to them that sit in darkness: Show yourselves. They shall feed in the ways, and their pastures shall be in all high places.

They shall not hunger nor thirst, neither shall the heat nor the sun smite them; for he that hath mercy on them shall lead them, even by the springs of water shall he guide them (Isaiah 49:8-10).\index{covenant, promises}
\end{quotation}

\begin{quotation}
But, behold Zion hath said: The Lord hath forsaken me, and my Lord hath forgotten me - but he will show that he hath not.

For can a woman forget her sucking child, that she should not have compassion on the son of her womb? Yea, they may forget, yet will I not forget thee, O house of Israel.

Behold, I have graven thee on the palms of my hands; they walls\footnote{i.e. the walls of Jerusalem or Zion} are continually before me (Isaiah 49:14-16).
\end{quotation}

\begin{quotation}
For the mountains shall depart, and the hills be removed; but my kindness shall not depart from thee, neither shall the covenant of my peace be removed, saith the Lord that hath mercy on thee (Isaiah 54:10).
\end{quotation}

\begin{quotation}
Verily I say unto you, that my people shall know my name; yea, in that day they shall know that I am he that doth speak (3 Nephi 20:39).
\end{quotation}

\subsection{The Promised Gathering}

\begin{quotation}
And then, O house of Israel, behold, these shall come from far; and lo, these from the north and from the west; and these from the land of Sinim\footnote{some say this refers to China, others to Phoenicia, and still others to Egypt. However, there does seem to be consensus that this refers to people not of the house of Israel.}.

Sing, O heavens; and be joyful, O earth; for the feet of those who are in the east shall be established; and break forth into singing, O mountains; for they shall be smitten no more; for the Lord hath comforted his people, and will have mercy upon his afflicted (Isaiah 49:12-13).
\end{quotation}

\begin{quotation}
And thou shalt return unto the Lord thy God, and shalt obey his voice according to all that I command thee this day, thou and thy children, with all thine heart, and with all thy soul;

That then the Lord thy God will turn thy captivity, and have compassion upon thee, and will return and gather thee from all the nations, whither the Lord thy God hath scattered thee. 

If any of thine be driven out unto the outmost parts of heaven, from thence will the Lord thy God gather thee, and from thence will he fetch thee:

And the Lord thy God will bring thee into the land which thy fathers possessed, and thou shalt possess it; and he will do thee good, and multiply thee above thy fathers.

And the Lord thy God will circumcise thine heart, and the heart of thy seed, to love the Lord thy God with all thine heart, and with all thy soul, that thou mayest live (Deuteronomy 30:3-6).
\end{quotation}

\section{Covenant keepers}

\paragraph{And they shall be mine}
Then they that feared the Lord spake often one to another: and the Lord hearkened, and heard it, and a book of rememberance was written before him for them that feared the Lord, and that thought upon his name. And they shall be mine, saith the Lord of hosts, in that day when I make up my jewels; and I will spare them, as a man spareth his own son that serveth him. (Malachi 3:16-17)

\paragraph{Yet I will own them}
Verily I say unto you, concerning your brethren who have been cast out from the land of their inheritance - I, the Lord, have suffered the affliction to come upon them, wherewith they have been afflicted, in consequence of their transgressions;\footnote{Transgressions do not change God's promise to save them in the end.} Yet, I will own them, and they shall be mine in that day when I shall come to make up my jewels. (D\&C 101:1-3)

\paragraph{I will search them, seek them, gather them, and feed them}
For thus saith the Lord God; behold, I, even I, will both search my sheep, and seek them out. As a shepherd seeketh out his flock in the day that he is among his sheep that are scattered; so will I seek out my sheep, and will deliver them out of all places where they have been scattered in the cloudy and dark day.

And I will bring them out from the people, and gather them from the countries, and will bring them to their own land, and feed them upon the mountains of Israel by the rivers, and in all the inhabited places of the country. 

I will feed them in a good pasture, and upon the high mountains of Israel shall their fold be: there shall they lie in a good fold, and in a fat pasture shall they feed upon the mountains of Israel.

I will feed my flock, and I will cause them to lie down, saith the Lord God. I will seek that which was lost, and bring again that which was driven away, and will bind up that which was broken, and will strengthen that which was sick: but I will destroy the fat and the strong; I will feed them with judgment. (Ezekiel 34:11-16)

\paragraph{He loveth those who will have him to be their God}
Behold, the Lord hath created the earth that it should be inhabited; and he hath created his children that they should possess it.

And he raiseth up a righteous nation, and destroyeth the nations of the wicked.

And he leadeth away the righteous into precious lands, and the wicked he destroyeth, and curseth the land unto them \textit{for their sakes}.

And he loveth those who will have him to be their God. Behold, he loved our fathers, and he covenanted with them, yea, even Abraham, Isaac, and Jacob; and he remembered the covenants which he had made; wherefore, he did bring them out of the land of Egypt.

And he did straighten them in the wilderness with his rod; for they hardened their hearts, even as ye have; and the Lord straightened them because of their iniquity. He sent fiery flying serpents among them; and after they were bitten he prepared a way that they might be healed; and the labor which they had to perform was to look; and because of the simpleness of the way, or the easiness of it, there were many who perished. (1 Nephi 17:36-38, 40-41)

\section{Covenant breakers}

\paragraph{Yet I loved Jacob and I hated Esau}
I have loved you, saith the Lord. Yet ye say, Wherein hast thou loved us? Was not Esau Jacob's brother? saith the Lord: yet I loved Jacob, And I hated Esau,\footnote{I believe the principle taught here is that God reserves unique love and blessings for His covenant children.} and laid his mountains and his heritage waste for the dragons of the wilderness. (Malachi 1:2-3)

\paragraph{Question about the covenant}
God established His covenant with Jacob and Esau's parents (Isaac and Rebeccah) and grandparents (Abraham and Sarah). Esau sold his birthright (Genesis 25:29-34) to Jacob thereby breaking the covenant. If both Esau and Isaac would have been righteous, would both have been included in the covenant family? Since God is no respecter of persons, I think the answer must be 'yes'.

\paragraph{The people against whom the Lord hath indigation forever}
Whereas Edom\footnote{Edom refers to the nation that came from Esau.} saith, We are impoverished, but we will return and build the desolate places; thus saith the Lord of hosts, They shall build, but I will throw down;\footnote{The descendants of Esau didn't make a covenant, but they appear to be cursed, as the descendants of Laman and Lemuel were.} and they shall call them, The border of wickedness, and, The people against whom the Lord hath indignation for ever.\footnote{Indignation forever? Is there a way for the Edomites to be redeemed?} (Malachi 1:4)

\paragraph{Ye offer polluted bread}
Ye offer polluted bread upon mine alter; and ye say, Wherein have we polluted thee? In that ye say, The table of the Lord is contemptible. And if ye offer the blind for sacrifice, is it not evil? and if ye offer the lame and sick, is it not evil? offer it now unto thy governor; will he be pleased with thee, or accept thy person? saith the Lord of hosts. (Malachi 1:7-8)

\paragraph{But now entreat God's favor}
But now entreat God's favor, That he may be gracious to us. While this is being done by your hands, will He accept you favorably? Says the Lord of hosts. (Malachi 1:9 New KJV)

\paragraph{Neither will I accept an offering at your hand}
Who is there even among you that would shut the doors for nought? neither do ye kindle fire on mine alter for nought. I have no pleasure in you, saith the Lord of hosts, neither will I accept an offering at your hand. (Malachi 1:10)

"Oh that one of you would shut the temple doors, so that you would not light useless fires on my alter! I am not pleased with you," says the LORD Almighty, "and I will accept no offering from your hands." (Malachi 1:10 NIV)

\paragraph{I will send a curse upon you}
And now, O ye priests, this commandment is for you. If ye will not hear, and if ye will not lay it to heart, to give glory unto my name, saith the Lord of hosts, I will even send a curse upon you, and I will curse your blessings: yea, I have cursed them already, because ye do not lay it to heart. (Malachi 2:1-2)

\paragraph{I will rebuke your seed}
Behold, I will [rebuke] your seed,\footnote{I'm not sure I know what this means. Do the children of covenant breakers suffer for the sins of their parents?} and spread dung upon your faces, even the dung of your solemn feasts; and one shall take you away with it. (Malachi 2:3)

\paragraph{But ye are departed out of the way}
But ye are departed out of the way; ye have caused many to stumble at the law; ye have corrupted the covenant of Levi, saith the Lord of hosts. Therefore have I also made you contemptible and base before all the people, according as ye have not kept my ways, but have been partial in the law. (Malachi 2:8-9)

\paragraph{Hath married the daughter of a strange god}
Judah hath dealt treacherously, and an abomination is committed in Israel and in Jerusalem; for Judah hath profaned the holiness of the Lord which he loved, and hath married the daughter of a strange god. The Lord will cut off the man that doeth this, the master and the scholar, out of the tabernacles of Jacob, and him that offereth an offering unto the Lord of hosts. (Malachi 2:11-12)

\paragraph{Return unto me and I will return unto you}
For I am the Lord, I change not; therefore, ye sons of Jacob are not consumed. Even in the days of your fathers ye are gone away from mine ordinances, and have not kept them. Return unto me, and I will return unto you, saith the Lord of hosts. (Malachi 3:6-7)

\paragraph{he did straighten them}
And he did straighten them in the wilderness with his rod; for they hardened their hearts, even as ye have; and the Lord straightened them because of their iniquity. He sent fiery flying serpents among them; and after they were bitten he prepared a way that they might be healed; and the labor which they had to perform was to look; and because of the simpleness of the way, or the easiness of it, there were many who perished. (1 Nephi 17:41)

\paragraph{As for those who are at Jerusalem}
And as for those who are at Jerusalem, saith the prophet, they shall be scourged by all people, because they crucify the God of Israel, and turn their hearts aside, rejecting signs and wonders, and the power and glory of the God of Israel.

And because they turn their hearts aside, saith the prophet [Zenos], and have despised the Holy One of Israel, they shall wander in the flesh, and perish, and become a hiss and a byword, and be hated among all nations (1 Nephi 19:13-14).

\paragraph{I will not cast them away}
Leviticus 26:14-18, 21, 23-24, 27-28, 41-46
The land also shall be left of them...

\paragraph{Then will he remember the covenants which he made with their fathers}
Nevertheless, when that day cometh, saith the prophet [Zenos], that they no more turn aside their hearts against the Holy One of Israel, then will he remember the covenants which he made to their fathers.

Yea, then will he remember the isles of the sea; yea, and all the people who are of the house of Israel, will I gather in, saith the Lord, according to the words of the prophet Zenos, from the four quarters of the earth.

Yea, and all the earth shall see the salvation of the Lord, saith the prophet; every nation, kindred, tongue and people shall be blessed (1 Nephi 19:15-17).\footnote{Nephi explains in the next two verses that he wrote these things specifically to \textit{his people} and generally to \textit{all the house of Israel} that he might persuade them to \textit{remember} the Lord their Redeemer.}

\paragraph{I will defer my anger}
Nevertheless\footnote{even though you have been unfaithful}, for my name's sake will I defer my anger, and for my praise will I refrain from thee, that I cut thee not off.

For, behold, I have refined thee, I have chosen thee in the furnace of affliction. (1 Nephi 20:9-10, Isaiah 48:9-10).

\section{Gentiles}

\paragraph{My name shall be great among the Gentiles}
For from the rising of the sun even unto the going down of the same my name shall be great among the Gentiles; and in every place incense shall be offered unto my name, and a pure offering: for my name shall be great among the heathen, saith the Lord of hosts. (Malachi 1:11)

\paragraph{The Lord esteemeth all flesh in one}
And now, do ye suppose that the children of this land, who were in the land of promise, who were driven out by our fathers, do ye suppose that they were righteous? Behold, I say unto you, Nay. Do ye suppose that our fathers would have been more choice than they if they had been righteous? I say unto you, Nay. Behold, the Lord esteemeth all flesh in one; he that is righteous is favored of God. But behold, this people had rejected every word of God, and they were ripe in iniquity; and the fulness of the wrath of God was upon them; and the Lord did curse the land against them, and bless it unto our fathers; yea, he did curse it against them unto their destruction, and he did bless it unto our fathers unto their obtaining power over it. (1 Nephi 33-35)

\section{Born in the Covenant}

There is such a thing as being born in the covenant, but the concept seems to be bigger than how we've traditionally thought of it in the church. For example, in the church we teach that children born to parents that are sealed together in the temple are born in the covenant. The concept also exists in the temple ordinance where a child is sealed to his or her parents.

However, Nephi teaches his brothers that being a literal descendant of Israel is also being born in the covenant. 

\begin{quotation}
And at that day shall the remnant of our seed know that they are of the house of Israel, and that they are the covenant people of the Lord; and then shall they know and come to the knowledge of their forefathers, and also to the knowledge of the gospel of their Redeemer, which was ministered unto their fathers by him; wherefore, they shall come to the knowledge of their Redeemer and the very points of his doctrine, that they may know how to come unto him and be saved. (1 Nephi 15:14)
\end{quotation}

\section{Scattering and Gathering of Israel}

Relevant scriptures:
\begin{enumerate}
  \item 1 Nephi 10
  \item 1 Nephi 15
\end{enumerate}

\subsection{Promises to Abrahamic}

Lehi taught that the Abrahamic covenant would be fulfilled in the latter days, specifically, Nephi referred to the part that says "In thy seed shall all kindreds of the earth be blessed." Nephi used this teaching as an example for Laman and Lemuel that Lehi was not speaking of their seed alone when he likened the house of Israel to an olive tree and about the scattering and gathering of Israel. In teaching this, Nephi also teaches that the Lord's covenant with Abraham means that God will \textit{gather} the house of Israel in the latter days.

\begin{quotation}
Wherefore our father hath not spoken of our seed alone, but also of all the house of Israel, pointing to the covenant which should be fulfilled in the latter days; which covenant the Lord made to our father Abraham, saying: In thy seed shall all the kindreds of the earth be blessed. (1 Nephi 15:18)
\end{quotation}

\section{Abrahamic Covenant}
Relevant scriptures:
\begin{itemize}
  \item Genesis 17:1-10, 15-16, 19 (Abraham and Sarah)
  \item Genesis 26:1-5, 24 (Isaac and Rebekah)
  \item Leviticus 26:42
  \item Acts 3:25
\end{itemize}
