\chapter{Doctrine of Jesus Christ}

\section{Definitions}

\begin{quotation}
The steps God revealed we must take to follow Jesus and receive eternal life are called the doctrine of Christ. They include "faithin Jesus Christ and His Atonement, repentance, baptism [into the Church of Jesus Christ of Latter Day Saints], receiving the gift of the Holy Ghost, and enduring to the end." (Do You Know Why I as a Christian Believe in Christ?, Ahmad S. Corbitt, April 2023 General Conference)
\end{quotation}

As he was explaining Lehi's teachings about the scattering and gathering of Israel to Laman and Lemuel, Nephi explains the basic notion of the doctrine of Christ. He says of their seed in the latter days

\begin{quotation}
wherefore, they shall come to the knowledge of their Redeemer and the very points of his doctrine, that they may know how to come unto him and be saved. (1 Nephi 15:14)
\end{quotation}

\section{Promises}

\begin{quotation}
The pure doctrine of Christ is powerful. It changes the life of everyone who understands it and seeks to implement it in his or her life. The doctrine of Christ helps us find and stay on the covenant path. Staying on that narrow but well-defined path will ultimately qualify us to receive all that God has. (Pure Truth, Pure Doctrine, and Pure Revelation, Russell M. Nelson, October 2021 General Conference)
\end{quotation}

\section{Scriptures}

\subsection{Lehi's Vision 1 Nephi 8-10}

\paragraph{The Fruit of the Tree of Life}
Lehi sees the tree of Life in his vision. He describes the fruit of the tree of life, and it's effect on him, like this:

\begin{quotation}
And it came to pass that I beheld a tree, whose fruit was desirable to make one happy.

And it came to pass that I did go forth and partake of the fruit thereof; and I beheld that it was most sweet, above all that I ever before tasted. Yea, and I beheld that the fruit thereof was white, to exceed all the whiteness that I had ever seen.

And as I partook of the fruit thereof it filled my soul with exceedingly great joy; (1 Nephi 8:10-12)
\end{quotation}

\paragraph{Lehi's teachings - 1 Nephi 10}
In response to his vision Lehi teaches his family about the Doctrine of Christ. His teachings included:

\begin{itemize}
  \item{God would raise up a prophet among the Jews who would be the Messiah or Savior of the World (v4).}
  \item{The Messiah is also the Redeemer of the world (v5).}
  \item{All mankind is in a fallen state (v6).}
  \item{The Messiah would be baptized with water (v9).}
  \item{The Lamb of God would take away the sins of the world (v10).}
  \item{The Jews would dwindle in unbelief and kill the Messiah and He would rise from the dead and would manifest Himself to the Gentiles (v11).}
\end{itemize}

\subsection{Nephi's vision of the Tree of Life}

Nephi desired to see the things his father saw (1 Nephi 10:17) and as a result the Lord showed him the things his father saw, but in segments with interaction and explanation in between each part. The significant parts of what Nephi saw include:

\begin{itemize}
  \item{The Son of God would be born to a mortal mother and the Eternal Father. The angel called this the condescension of God (v12-21).}
  \item{The condescension of God was directly related to the Tree of Life (v21-23).}
  \item After seeing the the Lamb of God born on earth, Nephi connected the Love of God to the Tree of Life (v22).
  \item The fountain of living waters and the tree of life are synonymous. The fountain of living waters is also a representation of the love of God (v25).
  \item Nephi seems to use the phrase love of God when he sees that God has prepared a way to save His children (v22, 25).
  \item The angel talks a second time about the condescension of God, this time before Nephi sees John baptize Jesus (v26) and Jesus' mortal ministry (v27-33).
\end{itemize}

\subsection{The Role of the Church in the Doctrine of Christ}

\begin{quotation}
And he gave [the Church organization] for the perfecting of the Saints, for the work of the ministry, for the edifying of the body of Christ: Till we all come in the unity of the faith, and of the knowledge of the Son of God, unto a perfect man, unto the measure of the stature of the fulness of Christ... (Ephesians 4:11-13).
\end{quotation}

\subsection{The fullness of the gospel and the doctrine of Christ}
When Nephi is explaining Lehi's vision to Laman and Lemuel (1 Nephi 15:12-14), he seems to equate the fulness of the gospel to two things:
\begin{itemize}
  \item authority to perform all ordinances necessary for salvation
  \item understanding of what the doctrine of Christ is
\end{itemize}

\begin{quotation}
And at that day shall the remnant of our seed know that they are of the house of Israel, and that they are the covenant people of the Lord; and then shall they know and come to the knowledge of their forefathers, and also to the knowledge of the gospel of their Redeemer, which was ministered unto their fathers by him; \emphasis{wherefore, they shall come to the knowledge of their Redeemer and the very points of his doctrine, that they may know how to come unto him and be saved.} (1 Nephi 15:14)
\end{quotation}
