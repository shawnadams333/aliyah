\chapter{Doctrine of Jesus Christ}

\section{Lehi's Vision 1 Nephi 8-10}

\paragraph{The Fruit of the Tree of Life}
Lehi sees the tree of Life in his vision. He describes the fruit of the tree of life, and it's effect on him, like this:

\begin{quotation}
And it came to pass that I beheld a tree, whose fruit was desirable to make one happy.

And it came to pass that I did go forth and partake of the fruit thereof; and I beheld that it was most sweet, above all that I ever before tasted. Yea, and I beheld that the fruit thereof was white, to exceed all the whiteness that I had ever seen.

And as I partook of the fruit thereof it filled my soul with exceedingly great joy; (1 Nephi 8:10-12)
\end{quotation}

\paragraph{Lehi's teachings - 1 Nephi 10}
In response to his vision Lehi teaches his family about the Doctrine of Christ. His teachings included:

\begin{itemize}
  \item{God would raise up a prophet among the Jews who would be the Messiah or Savior of the World (v4).}
  \item{The Messiah is also the Redeemer of the world (v5).}
  \item{All mankind is in a fallen state (v6).}
  \item{The Messiah would be baptized with water (v9).}
  \item{The Lamb of God would take away the sins of the world (v10).}
  \item{The Jews would dwindle in unbelief and kill the Messiah and He would rise from the dead and would manifest Himself to the Gentiles (v11).}
\end{itemize}

\paragraph{Nephi's vision of the Tree of Life}
Nephi desired to see the things his father saw (1 Nephi 10:17) and as a result the Lord showed him the things his father saw, but in segments with interaction and explanation in between each part. The significant parts of what Nephi saw include:

\begin{itemize}
  \item{The Son of God would be born to a mortal mother and the Eternal Father. The angel called this the condescension of God (v12-21).}
  \item{The condescension of God was directly related to the Tree of Life (v21-23)}
\end{itemize}
